\newpage
\section{Vorwort}
\subsection{Themenbegründung}
Die Entscheidung für das Brauen kam nicht sofort. Andere Ideen
 wie ein intelligenter Spiegel für das Badezimmer oder generell 
 ein Smarthome kamen uns in den Sinn. Wir hielten dies jedoch nicht
  für interessant genug. Als Gruppe beschlossen wir, ein Thema zu behandeln, das nicht direkt mit IT zu tun hat. Wir haben versucht, 
   ein Thema zu finden, das uns Spass macht, das neu und interessant ist und einen sozialen Aspekt hat.
Wir waren uns über das Brauen einig,
 da es etwas Neues ist. Etwas, das nicht jeder macht, 
 und das einen sozialen Aspekt hat, wenn man Brauer trifft und unser Bier mit Freunden und Familie testet.
 \subsection{Persönlicher Bezug - Janik Spies}
 Ich hatte keine wirkliche Idee für das Thema der VA. Callum ist dann mit der Idee gekommen die VA über die Bierbrauerei zu machen. Das hat mich angesprochen, da ich dieses Thema schon einmal privat angeschaut habe und auch schon einige Informationen über das Thema gesammelt habe. Ursprünglich bin ich auf das Thema Bierbrauerei gekommen, da ich öfters im Ausgang war und dort schon verschiedenste Biersorten versucht habe. Dabei hat mich aber keine richtig überzeugt. Deswegen habe ich auch schon dann überlegt, ob es sich nicht lohnen würde ein eigenes Bier zu Brauen. Schnell kam auch die Idee damit Geld zu verdienen. Im Idealfall würden wir beim Verkauf von Bier unseren Eigenkonsum decken und somit gratis unser Bier geniessen können. Auch das Bierbrauen in Grossbetrieben hat mich brennend interessiert, 
 da das zwei Leidenschaften von mir, Bier und Unternehmertum, zusammen verbindet.
\subsection{Persönlicher Bezug - Callum Stringer}
Mein persönlicher Bezug zur Brauerei war mir ungemein wichtig. Ich habe vorher zu Hause ein wenig gebraut. 
Ich war sehr begeistert, meine Vorkenntnisse in einem Schulprojekt einzusetzen. Obwohl ich schon Erfahrung habe, habe ich schon eine ganze Weile nichts gebraut.
 Ich hoffe, dass dieses Projekt mein Interesse und meine Motivation, noch etwas Bier zu brauen, wieder wecken wird.
 \subsection{Persönlicher Bezug - Fabrizio Franco}
 Meine Partner und ich haben uns zusammen für das Thema 'Bier brauen' entschieden. Ich war für dieses Thema, weil ich selber
  gerne Bier trinke aber nicht genau weiss, was in Bier enthalten ist oder wie überhaupt die Herstellung von Bier erfolgt. 
  Dazu kommt noch, dass mein Kollege Callum uns gut helfen kann, da er schon mal selber zuhause Bier gebraut hat und uns mit diesem 
  Experiment gut unterstützen kann. Es wundert mich zu wissen ob die Produktion von unterschiedlichen Sorten gleich abläuft oder 
  ob es bei bestimmten Sorten Abweichungen gibt. Mich interessiert auch zu wissen welche Unterschiede es bei der Herstellung von Bier
   zwischen Grossbrauereien wie Feldschlösschen und Kleinbrauereien gibt und wie die Preise für die verschiedenen Biersorten zustande 
   kommen. Hier in der Schweiz gibt es viele Kleinbrauereien und mich wundert es ob diese ein gewinnbringendes Geschäft oder eher eine
    Freizeit Beschäftigung sind. 
 Ich möchte mit diesem Projekt mein Wissen bezüglich Bier massiv erweitern da ich nur sehr weniger Erfahrung (abgesehen vom trinken) damit habe.
\subsection{\LaTeX\ }
Der Entschluss, \LaTeX\  als Werkzeug zur Dokumentenvorbereitung zu verwenden, wurde recht schnell gefasst. Wir haben \LaTeX\  bereits in
 der PVA verwendet. Es hat uns bei der Formatierung wirklich sehr geholfen und die Dinge einfach gehalten,
 so dass wir uns auf das Projekt konzentrieren konnten und uns keine Gedanken über das Herumfummeln mit Word machen mussten.
\\
Wir benutzten ein Git-Repository, um unsere Arbeit zu verfolgen, was es sehr einfach machte, als Team zusammenzuarbeiten. 
\subsection{Zusammenhang zwischen Klassen- und Gruppenthema}
Es war eine Herausforderung, eine Verbindung zwischen dem Bierbrauen und dem Thema 'Neue Wege' zu finden, aber man könnte es so betrachten, dass diese ganze Erfahrung, das Bierbrauen zu lernen, für uns ein neues Abenteuer und eine neue Erfahrung war. Man könnte also sagen, es war ein neuer Weg für uns als Gruppe.
\newpage
\subsection{Ziel und Zielbegründung}
\textbf{Finanziell lohnenswertes Bierbrauen}\\
Mit diesem Ziel wollten wir herausfinden, ob es sich finanziell lohnen würde sein eigenes Bier zu brauen. Dazu gehören die Herstellungskosten und die eventuellen Erträge durch den Bierverkauf. Unser Ziel dabei ist, die Kosten vom eigenen Bierkonsum so niedrig wie möglich zu halten.
\\\\\textbf{Kommerzielles Bierbrauen}\\
Mit diesem Ziel wollten wir herausfinden, wie das Bier in Grossbetrieben hergestellt wird und welche Unterschiede es zu den Heimbrauereien es gibt.
\\\\\textbf{Unterschied Alkoholfrei und Alkoholhaltig}\\
Bei dem Thema Bier hört man sehr oft das Thema alkoholfreies Bier. Uns hat interessiert, ob die Menschen einen Unterschied zwischen dem alkoholfreiem und dem alkoholhaltigen Bier erkennen. 
\\\\\textbf{Geschmacksergebnis bei der Eigenbrauerei}\\
Können wir ein geschmacklich vergleichbares Bier erstellen oder merken die Menschen den Unterschied zwischen dem selbst gebrauten Bier und dem gekauften Bier.
\\\\\textbf{Geschmackstest von Personen}\\
Können Personen in unserem Bekanntenkreis verschiedene Biersorten voneinander unterscheiden oder können sie das Bier nur am Etikett erkennen?

\newpage
\section{Einleitung}
\subsection{Einführung}
Brauen ist ein grosses Thema, wie man von dem Mindmap ablesen kann. 
Aber wir wollten uns auf einige wenige Themen konzentrieren.
Zunächst einmal haben wir uns für das Brauen selbst entschieden, 
wir wollten diese Erfahrung selbst durchgehen. Wir haben uns entschieden,
unser eigenes Bier mit Hilfe eines Bierkastens zu brauen. Dazu beschlossen wir auch,
eine Verkostungsveranstaltung durchzuführen, um zu sehen, ob unser Bier gut ist. 
Das zweite Thema, das wir uns anschauen wollten, ist, ob es sich lohnt, ein eigenes Bier zu brauen. Dazu gehörten die Finanzen des Bierbrauens und ob es billiger geht. Dazu gehört auch, ob das Bier gut genug schmeckt. Es könnte billig sein, aber es könnte sich nicht lohnen, wenn der Geschmack schlecht ist.
\\
Zu Referenzzwecken wird 'Vertiefungsarbeit' im gesamten Umfang dieses Dokuments als VA gekürzt
\newpage
\subsection{Mindmap}
Das Mindmap ist in den Beilagen ersichtlich, wir setzten uns hin und schrieben jede Idee auf, die uns in dieser Zeit in den Sinn kam. Danach nutzten wir diese als Planungshilfe und entschieden, welche Informationen und Themen wir untersuchen wollen.
Die Hauptthemen, für die wir uns entschieden haben, sind die folgenden:
\begin{itemize}
   \item Umgebung - Besuch der Brauereien, Interviews/Besuch in der Brauereien
   \item Biersorten - Beim Heimbrauen entschieden wir uns für spezifische Biersorten
   \item Herstellung
   \item Finanziell
   \item Design
\end{itemize}
Die folgenden Ideen wurden aufgrund der zeitlichen Beschränkungen dieses Projekts nicht geprüft:
\begin{itemize}
   \item Wirtschaft sowie Kultur
   \item Alkohol
   \item Umfrage
   \item Geschichte
\end{itemize}



