\subsection{Schutz Konzept}
Das vorliegende Schutzkonzept beschreibt, welche Vorgaben bei der Degustation einzuhalten sind, solange aufgrund COVID-19 besondere Hygiene- und Abstandsregeln geltend sind. Das Schutzkonzept richtet sich an das Personal, sowie auch die Gäste der Bier Degustation und dient der Festlegung von besonderen Schutzmassnahmen, die von allen eingehalten werden müssen. Ausnahmen können mit Begründung und vorlegen von Arztzeugnis getätigt werden.
Diese Massnahmen sorgen dafür, dass sich das Coronavirus im Zuge der Degustation nicht weiter ausbreitet. 
\\
\textbf{Personal:}
\begin{itemize}
    \item Alle Bierflaschen werden vor dem Gebrauch von aussen desinfiziert.
    \item Tische werden gereinigt.
    \item Desinfektionsmittel und Masken werden vom Personal bereitgestellt.
    \item Abfall mit geschlossenem Deckel wird bereitgestellt.
    \item Fenster bleiben offen.
    \item Es müssen Handschuhe getragen werden.
\end{itemize}


\textbf{Gäste:}

\begin{itemize}
    \item 1,5 Meter Abstand.
    \item Maskenpflicht bis zum Degustationsstand.
    \item Kein Körperkontakt
    \item Bei Symptomen zu Hause bleiben.
\end{itemize}

Dieses Konzept wurde Stand 09.11.2020 verfasst.
Es kann wegen der Corona Situation zu Änderungen kommen. In dieser Situation wird das Schutzkonzept angepasst.
Infos zu Änderungen holen wir direkt vom BAG \footcite[Bundesamt für Gesundheit BAG]{bag.admin}

\newpage
\textbf{Durchführung}\\
Dank der ausführlichen Planung hatten wir einen entscheidenden Vorteil bei der Ausführung der Degustation.
 Das Schutzkonzept konnte gut eingehalten werden. Jedoch mussten wir nicht alle darauf aufgeführten Bedingungen ausführen
 , da wir im Familien- und Kollegenkreis degustiert haben und dies jeweils mit Einzelpersonen durchgeführt haben. Diese Variante
hat uns zwar mehr Zeit gekostet, dafür hatten wir eine grössere Sicherheit im Bezuge zum Corona Virus und wir konnten gleich persönlich 
auf Fragen und Kommentare von den Testpersonen eingehen. Das hatte natürlich zur Folge, dass wir mehr darauf achten mussten, dass wir uns nicht
verplappern und die Testperson erraten konnte welches Bier, das unsere ist. Beim Degustieren ist uns aufgefallen, dass unser Weizenbier gleichermassen
bewertet wurde wie das Weizenbier des Konkurrenten. Auch das Alkoholfreie wurde gut bewertet. Anders hat es jedoch bei dem Cherry Ale
ausgesehen. Dort wurden teils sogar die Gesichter verzogen. Durch das Befragen ausserhalb des Auswertungsblattes hat sich jedoch
herauskristallisiert, dass die Befragten Cherry Ale im Allgemeinen nicht besonders gerne haben. Somit ist es auch klar, 
dass diese Biersorte schlechter bewertet wurde. Während der Durchführung konnte jedoch kein grosser Unterschied zwischen den Biersorten
festgestellt werden. Das konnte erst bei der Auswertung genau analysiert werden.
