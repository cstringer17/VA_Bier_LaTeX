\subsection{Schlusswort}
\subsubsection{Persönliche Reflexion - Fabrizio Franco}
\subsubsection{Persönliche Reflexion - Callum Stringer}
Ich habe diese Arbeit sehr genossen. Ich habe mich in der Vergangenheit immer für das Brauen interessiert. Es ist sehr befriedigend, draußen zu sitzen und sein selbstgebrautes Bier zu trinken, nachdem man einen Tag Arbeit und wochenlanges Warten hinter sich gebracht hat.
Ich genoss die Erfahrung, als ich es alleine machte, aber dieses Hobby mit einigen Freunden zu teilen, hat mir sehr viel Spaß gemacht.
Ich scheine im Laufe des Prozesses eine Menge Dinge zu lernen. Viele kleine Tipps und Tricks, an die ich mich in Zukunft erinnern werde.
Zum Beispiel ist es wichtig, etwas Papier unter den Zapfhahn des Fasses zu legen, da ich früher klebriges Bier vom Boden reinigen musste,
nachdem ich es tagelang nicht kontrolliert hatte.
Eine andere Sache ist es, vorbereitet zu sein und alles bereit zu haben, wie eine Art Mise en Place. Das macht es später einfacher, da man
nicht alles suchen muss.
Ein weiterer Punkt ist, dass das Fass vor der Herstellung des Bieres leicht erscheinen mag, aber 20 Liter Bier sind ziemlich schwer,
und es in der Küche zuzubereiten und es dann vorsichtig nach unten tragen zu müssen, war nicht die einfachste Sache.

Abgesehen vom Bier lief das Projekt als Ganzes gut, wir hatten einige Probleme mit dem Zeitmanagement in der PVA, aber dieses Mal haben wir uns einen Moment Zeit genommen, um anzuhalten und den gesamten Prozess zu planen. Dafür schufen wir ein Trello-Board. Mit diesem kann man To-Dos hinzufügen und den Fortschritt bei bestimmten Aufgaben verfolgen. Es war sehr nützlich, einen Überblick darüber zu behalten, was getan werden musste.

\subsubsection{Persönliche Reflexion - Janik Spies}
