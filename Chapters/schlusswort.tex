\subsection{Schlusswort}
\subsubsection{Inhaltliche Schlusswort}
Während dieses Projekts haben wir eine Menge über das Brauen gelernt. Wir haben den Prozess des Bierbrauens selbst durchlaufen. Wir fanden auch heraus, wie sehr das Brauen ein großes Thema ist und wie tief man in die Tiefe gehen kann. Wir fanden es interessant, Bier aus einer kleinen Wanne in einem Haus zu brauen. Es war eine Erfahrung, die sehr interessant und voller lustiger Momente war. 

Unser Bier kam recht gut heraus.
 Als Gruppe haben wir beschlossen, dass uns das Bier schmeckt. 
 Was für uns aber auch wichtig war, war, wie andere Leute darüber dachten.
  Unsere Verkostungsveranstaltung verlief gut, und wir waren mit den Ergebnissen zufrieden.
   Wir hätten nie erwartet, dass es das am besten schmeckende Bier sein würde, aber dass die Leute 
   keine großen Probleme damit hatten, war für uns großartig.

\subsubsection{Zielkontrolle}
\textbf{Finanziell lohnenswertes Bierbrauen}\\
Wir konnten im Kapitel 4.1 genau sehen, dass es sich finanziell nicht lohnen würde sein eigenes Bier zu produzieren. Der Verkaufspreis müsste um weites höher sein als die der Marktgegner. Wir haben unser Ziel zwar erreicht, da wir vorerst nur herausfinden wollten, ob es sich lohnen würde. Aber wir konnten unser Unterziel, es finanziell Lohnenswert zu machen nicht erreichen.
\\
\textbf{Kommerzielles Bierbrauen}\\
Leider konnten wir dieses Ziel bislang nicht erreichen, da uns alle Brauereien abgesagt haben oder sich nicht mehr gemeldet haben. Trotzdem bleiben wir offen und schauen, dass wir das Ziel eventuell noch bis zur Präsentation der VA erreichen können.
Unterschied Alkoholfrei und Alkoholhaltig
Durch die Degustation konnten wir dieses Ziel sehr leicht erreichen. Dabei wurden die Testpersonen jeweils gefragt, ob sie denken, dass das getestete Bier alkoholfrei oder alkoholhaltig ist. Dabei wurde schnell klar, dass die Mehrheit einen klaren Unterschied erkennt.
\\\newpage
\textbf{Geschmacksergebnis bei der Eigenbrauerei}\\
Auch dieses Ziel wurde in der Degustation erreicht. Dabei wurden verschiedene Biersorten, unter anderem auch unser selbst gebrautes Bier, miteinander verglichen und bewertet. Unser Bier wurde dabei immer ein wenig schlechter bewertet als die gekauften Biersorten, jedoch konnte es trotzdem sehr gut mithalten.
\\
\textbf{Geschmackstest von Personen}\\
Dieses Ziel wurde ebenfalls bei der Degustation erreicht, indem wir die Befragten mit der Auswahl der Biersorte untersuchten. Dabei kam heraus, dass es einige gibt, für die es sehr schwer war das Weizen- und Hopfenbier zu unterscheiden. Somit kann man sagen, dass es in Abhängigkeit zu der Biersorte viele Leute gibt, die Biersorten nicht wirklich unterscheiden können.
\\

\subsubsection{Persönliche Reflexion - Fabrizio Franco}
Dieses Projekt war wahrscheinlich eins der besten und lustigsten, an denen ich mich je selber beteiligen durfte. Ich freue mich im Nachhinein sehr, diesen Auftrag, mit Callum und Janik ausgeführt zu haben. Wir hatten viel Spass bei der Zubereitung des Biers und beim Planen des ganzen Auftrags. Ich habe viele neue Kenntnisse im Zusammenhang mit der Herstellung von Bier und den verschiedenen Brauarten gemacht. Ich selber war noch nie in einer Brauerei und wusste vor diesem Projekt gar nicht aus was Bier besteht und wie dieses überhaupt zustande kommt. Durch diesem Auftrag habe ich eine Art des Bierbrauens kennengelernt und konnte mein Wissen erweitert und habe dazu gemerkt, dass es nicht so schnell geht, wie ich es mir erhofft hatte, das frisch gemachte Bier zu trinken. Es musste mehrere Wochen lang ruhen, damit das Bier gut ist und genossen werden kann. Probleme tauchten eigentlich keine auf. Das einzige was man als "kleines Problem" betrachten könne, ist dass das Bier beim Öffnen der Flasche überschäumt. Doch trotz dessen kann man es, nach einigen Minuten warten, genussvoll trinken und geniessen. Die Degustation mit meinen Freunden durchführen gedurft zu haben, war für mich eines der besten Erlebnissen dieser Vertiefungsarbeit und es hat mich glücklich gemacht, dass das Bier ihnen gefallen hat. Im Grossen und Ganzen hat mir diese Arbeit sehr viel Spass bereitet und ich denke
 wir haben ein tolles Projekt auf die Reihe bekommen und wir sind glücklich mit unserem Ergebnis und dem Bier.
\subsubsection{Persönliche Reflexion - Callum Stringer}
Ich habe diese Arbeit sehr genossen. Ich habe mich in der Vergangenheit immer für das Brauen interessiert. Es ist sehr befriedigend, draußen zu sitzen und sein selbstgebrautes Bier zu trinken, nachdem man einen Tag Arbeit und wochenlanges Warten hinter sich gebracht hat.
Ich genoss die Erfahrung, als ich es alleine machte, aber dieses Hobby mit einigen Freunden zu teilen, hat mir sehr viel Spaß gemacht.
Ich scheine im Laufe des Prozesses eine Menge Dinge zu lernen. Viele kleine Tipps und Tricks, an die ich mich in Zukunft erinnern werde.
Zum Beispiel ist es wichtig, etwas Papier unter den Zapfhahn des Fasses zu legen, da ich früher klebriges Bier vom Boden reinigen musste,
nachdem ich es tagelang nicht kontrolliert hatte.
Eine andere Sache ist es, vorbereitet zu sein und alles bereit zu haben, wie eine Art Mise en Place. Das macht es später einfacher, da man
nicht alles suchen muss.
Ein weiterer Punkt ist, dass das Fass vor der Herstellung des Bieres leicht erscheinen mag, aber 20 Liter Bier sind ziemlich schwer,
und es in der Küche zuzubereiten und es dann vorsichtig nach unten tragen zu müssen, war nicht die einfachste Sache.

Abgesehen vom Bier lief das Projekt als Ganzes gut, wir hatten einige Probleme mit dem Zeitmanagement in der PVA, aber dieses Mal haben wir uns einen Moment Zeit genommen, um anzuhalten und den gesamten Prozess zu planen. Dafür schufen wir ein Trello-Board. Mit diesem kann man To-Dos hinzufügen und den Fortschritt bei bestimmten Aufgaben verfolgen. Es war sehr nützlich, einen Überblick darüber zu behalten, was getan werden musste.
\newpage
\subsubsection{Persönliche Reflexion - Janik Spies}
Mir hat die VA sehr gut gefallen. Wie auch schon oben erwähnt habe ich einen engen Bezug zum Thema Bier und wollte auch unbedingt wissen, ob sich das Bierbrauen finanziell lohnen würde. Es hat mich jedoch sehr enttäuscht, dass wir vor allem aufgrund Corona keinen Besuch in einer Brauerei machen konnten. Dort hätten wir noch sehr viele Informationen bekommen, vor allem was den Bereich Verkauf angehen würde. Aber abgesehen von dem, hatten wir sehr viel Spass bei der VA und vor allem beim Bierbrauen und Testen. Es gab viele spannende und lustige Momente, auf die ich bei der Präsentation noch genauer eingehen werde. Auch mit dem Erreichen unserer Ziele bin ich sehr zufrieden.
