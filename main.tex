
\documentclass[12pt]{scrartcl} 


%----------------------------------------------------------------------------------------
%	PACKAGES AND OTHER DOCUMENT CONFIGURATIONS
%----------------------------------------------------------------------------------------

\usepackage{amsmath, amsfonts, amsthm} % Math packages

\usepackage{listings} % Code listings, with syntax highlighting

\usepackage[german]{babel} % English language hyphenation

\usepackage{graphicx} % Required for inserting images
\graphicspath{{Figures/}{./}} % Specifies where to look for included images (trailing slash required)

\usepackage{booktabs} % Required for better horizontal rules in tables

\numberwithin{equation}{section} % Number equations within sections (i.e. 1.1, 1.2, 2.1, 2.2 instead of 1, 2, 3, 4)
\numberwithin{figure}{section} % Number figures within sections (i.e. 1.1, 1.2, 2.1, 2.2 instead of 1, 2, 3, 4)
\numberwithin{table}{section} % Number tables within sections (i.e. 1.1, 1.2, 2.1, 2.2 instead of 1, 2, 3, 4)

\setlength\parindent{0pt} % Removes all indentation from paragraphs

\usepackage{enumitem} % Required for list customisation
\setlist{noitemsep} % No spacing between list items

%----------------------------------------------------------------------------------------
%	DOCUMENT MARGINS
%----------------------------------------------------------------------------------------

\usepackage{geometry} % Required for adjusting page dimensions and margins

\geometry{
	paper=a4paper, % Paper size, change to letterpaper for US letter size
	top=2.5cm, % Top margin
	bottom=3cm, % Bottom margin
	left=3cm, % Left margin
	right=3cm, % Right margin
	headheight=0.75cm, % Header height
	footskip=1.5cm, % Space from the bottom margin to the baseline of the footer
	headsep=0.75cm, % Space from the top margin to the baseline of the header
	%showframe, % Uncomment to show how the type block is set on the page
}

\usepackage[utf8]{inputenc} % Required for inputting international characters
\usepackage[T1]{fontenc} % Use 8-bit encoding
\usepackage{fourier} % Use the Adobe Utopia font for the document
\usepackage{sectsty} % Allows customising section commands

\sectionfont{\vspace{6pt}\centering\normalfont\scshape} % \section{} styling
\subsectionfont{\normalfont\bfseries} % \subsection{} styling
\subsubsectionfont{\normalfont\itshape} % \subsubsection{} styling
\paragraphfont{\normalfont\scshape} % \paragraph{} styling

%----------------------------------------------------------------------------------------
%	HEADERS AND FOOTERS
%----------------------------------------------------------------------------------------

\usepackage{scrlayer-scrpage} % Required for customising headers and footers

\ohead*{} % Right header
\ihead*{} % Left header
\chead*{} % Centre header

\ofoot*{\scriptsize Callum Stringer, Janik Spies und Fabrizio Franco} % Right footer
\ifoot*{\scriptsize Neue Wege - Bier brauen} % Left footer
\cfoot*{\pagemark} % Centre footer
\usepackage{pgf-pie}
\usepackage{pgfplots}

\linespread{1.241}

\usepackage{attachfile}

\usepackage{tablefootnote}

\usepackage{caption}
\newcommand{\source}[1]{\caption*{Quelle: {#1}} }


\usepackage[
backend=biber,
style=alphabetic,
sorting=ynt
]{biblatex}
\addbibresource{sample.bib}


\title{	
	\normalfont\small
	\textsc{Gibmit - VA - IBE17B}\\ 
	{Lehrperson - Herr Martin Meneghin}
	\vspace{25pt} 
	\rule{\linewidth}{0.5pt}\\
	\vspace{16pt} 
	{\huge Neue Wege - Bier brauen}\\ 
	\vspace{10pt} 
	\rule{\linewidth}{2pt}\\ 
	\author{\large Callum Stringer, Janik Spies und Fabrizio Franco} 
	\vspace{12pt} 
	\date{\small\today} 
}


\begin{document}

\maketitle
\thispagestyle{empty}
\newpage
\tableofcontents

\section{Vorwort}
\subsection{Themenbegründung}
\subsection{Persönliche Bezug}
\subsection{Ziel und Zielbegründung}

\section{Einleitung}
\subsection{Einführung}

\newpage


\subsection{Mindmap}


\subsection{Ziele} \label{introduction}

\section{Hauptteil}
\section{Bierbrau Vorgang}
\subsection{Bierarten und Brauarten}
Es gibt viele Möglichkeiten, Bier und noch mehr Biersorten zu brauen.
Bier zu brauen gibt es schon so lange, wie es Menschen gibt. Es gibt sogar Aufzeichnungen über alte Eygpten und Mesopotamien \footnote{
	Mesopotamien oder Zweistromland bezeichnet die Kulturlandschaft in Vorderasien, die durch die großen Flusssysteme des Euphrat und Tigris geprägt wird.
}.\\
Jedes Getreide, das bestimmte Zucker enthält, könnte fermentieren. Da es wilde Hefen in der Luft gibt.
Das bedeutet, dass selbst in der Zeit der Stämme, als sie mit der Landwirtschaft begannen, eine Art Bier hergestellt worden sein könnte.\\
Das meiste Bier wird aus einer Art Stärkequelle hergestellt, die fermentiert werden kann, Brauereihefe zur Herstellung und Einleitung der
Gärung und Aromatisierung wie Hopfen, um die Süße des Zuckers auszugleichen



\subsection{Bestellung}
Zur Vorbereitung des Bierbrauens benötigten wir Materialien.
Wir entschieden uns für die Verwendung von "Bier-Kits". Aus Zeitgründen war es nicht möglich,
das Bier auf traditionelle Weise zu brauen, da dies zu lange dauern würde.
Wir bestellten zwei verschiedene Kits, von denen das eine ein Cherry Ale und das andere ein belgisches Saisonbier war.
Wir haben uns für diese beiden Biere entschieden, zum einen, weil wir einen Grundtyp von Bier brauen wollten, zum anderen, weil wir etwas experimentelleres brauen wollten.  Das experimentelle ist das Cherry Ale. Wir hatten so etwas in der Schweiz noch nie so leicht erhältlich gesehen und wollten es unbedingt ausprobieren.
Der zweite Grund war die Verfügbarkeit. Wir hatten uns schon früher für Tyuoes entschieden, und sie waren nicht schnell verfügbar. Und wegen des Zeitdrucks bei diesem Projekt mussten wir das nehmen, was von Anfang an verfügbar war.

Wir hatten bereits Ausrüstung, um das Bier selbst zu brauen, wie Callum Stringer dies schon einmal getan hatte.
Unsere definitive Reihenfolge war wie folgt.

\textbf{Schon vorhandene Artikel}
\begin{itemize}
	\item Starter-Paket de Luxe für Bierkits - Preis CHF 79.00
\end{itemize}
\textbf{Neu bestellte Artikel}
\begin{itemize}
	\item Bügelflasche braun kompl. 0,5 L - Preis CHF 126.00
	\item Brewferm Bierkit Cherry Ale - Preis CHF 29.00
\end{itemize}
\subsection{Vorbereiten}
\subsubsection{Disinfektion}
Zunächst einmal mussten wir alles vollständig reinigen.
Dazu bereiteten wir eine Desinfektionslösung mit den im Bierkit enthaltenen Mischgranulaten vor.
Dies war wichtig, da wir nicht wollen, dass etwas Unbekanntes im Bier ist, wenn es gärt.
Danach konnten wir mit dem eigentlichen Brauen des Bieres fortfahren. \\

Die Desinfektion hat einige Zeit in Anspruch genommen. Wir haben das Bier im Haus von Fabrizio Francos Familie gebraut.
Und Platz zu finden, um die großen Eimer durchgehend zu reinigen,
war eine echte Herausforderung. Am Ende haben wir viele der Reinigungsarbeiten in der Badewanne durchgeführt.
Hier gab es auch einige andere Herausforderungen. Die Desinfektionslösung war nicht die neutralste Flüssigkeit.
Callum Stringers dazu zu bringen, die Hände jucken und zornig zu machen. Danach entschieden wir uns für die Verwendung von Reinigungshandschuhen.
Der Prozess verlief danach sehr viel reibungsloser


\subsubsection{Bierkit vorbereitung}
Zuerst mussten wir die Bausätze öffnen. Nach dem Öffnen mussten wir sie aufwärmen. Dazu legten wir
sie etwa 15 Minuten lang in eine mit Wasser gefüllte Pfanne bei schwacher Hitze. Sobald es warm war,
wurde es in die großen Braubehälter gegossen. Danach fügten wir das Wasser hinzu. Je nach Art des Bierkits
geschah dies in verschiedenen Schritten. Zuerst warmes Wasser, dann lauwarmes Wasser auf die Gesamtmenge.
Einige dieser Wasserchargen waren Zuckerlösungen. Da die Hefe Zucker zum Leben und Wachsen braucht.
Danach hatten wir zwei große Fässer des jungen Bieres.
Dieses musste vor der Abfüllung gelagert werden, wir legten es in den Kasten und ließen es dort für 10 Tage stehen.

\section{Finazialle Sicht}
\section{Degustation}


\section{Lernziel}

\section{Schlussbetrachtung}

\subsection{Schlusswort}
\subsubsection{Persönliche Reflexion - Fabrizio Franco}
\subsubsection{Persönliche Reflexion - Callum Stringer}
\subsubsection{Persönliche Reflexion - Janik Spies}

\nocite{*}
\printbibliography[title={Quellenverzeichnis}]
\addcontentsline{toc}{section}{Quellenverzeichnis}


\renewcommand{\listfigurename}{Abbildungsverzeichnis}
\renewcommand{\listtablename}{Tabellenverzeichnis}
\addcontentsline{toc}{section}{Abbildungsverzeichnis}
\addcontentsline{toc}{section}{Tabellenverzeichnis}


\vspace*{4em}\noindent
\hfill%
\begin{tabular}[t]{c}
	\rule{10em}{0.4pt} \\ Callum Stringer
\end{tabular}%
\hfill%
\begin{tabular}[t]{c}
	\rule{10em}{0.4pt} \\ Datum
\end{tabular}%
\hfill\strut

\vspace*{4em}\noindent
\hfill%
\begin{tabular}[t]{c}
	\rule{10em}{0.4pt} \\ Fabrizio Franco
\end{tabular}%
\hfill%
\begin{tabular}[t]{c}
	\rule{10em}{0.4pt} \\ Datum
\end{tabular}%
\hfill\strut

\vspace*{4em}\noindent
\hfill%
\begin{tabular}[t]{c}
	\rule{10em}{0.4pt} \\ Janik Spies
\end{tabular}%
\hfill%
\begin{tabular}[t]{c}
	\rule{10em}{0.4pt} \\ Datum
\end{tabular}%
\hfill\strut


\end{document}
